\begin{frame}
  \titlepage
\end{frame}


\begin{frame}\frametitle{Gianluca Della Vedova}
\begin{itemize}
\item
Elementi di Bioinformatica
\item
Ufficio U14-2041
\item
\url{https://gianluca.dellavedova.org}
\item
\url{https://elearning.unimib.it/course/view.php?id=19214}
\item
\url{gianluca.dellavedova@unimib.it}
\item\url{https://github.com/bioinformatica-corso/programmi-elementi-bioinformatica}
\item\url{https://github.com/bioinformatica-corso/lezioni}
\end{itemize}
\end{frame}

\begin{frame}[fragile]
\frametitle{Notazione}
\begin{itemize}
\item
\alert{simbolo}: $T[i]$\\
\item
\alert{stringa}: $T[1]T[2]\cdots T[l]$\\
\item
\alert{sottostringa}: $T[i:j]$\\
\item
\alert{prefisso}: $T[:j]=T[1:j]$\\
\item
\alert{suffisso}: $T[i:]=T[i:|T|]$
\item
\alert{concatenazione}: $T_{1}\cdot T_{2} = T_{1}T_{2}$
\end{itemize}
\end{frame}


\begin{frame}[fragile]
\frametitle{Pattern Matching}
\begin{block}{Problema}
\alert{Input}: testo $T=T[1]\cdots T[n]$, pattern $P=P[1]\cdots P[m]$, alfabeto $\Sigma$\\
\alert{Goal}: trovare \emph{tutte} le occorrenze di $P$ in $T$\\
\alert{Goal}: trovare tutti gli $i$ tale che $T[i]\cdots T[i+m-1]=P$
\end{block}
\begin{block}{Algoritmo banale}
\alert{Tempo}: $O(nm)$
\end{block}
\begin{block}{Lower bound}
\alert{Tempo}: $O(n+m)$
\end{block}
\end{frame}

\begin{frame}
\frametitle{Bit-parallel}
\begin{block}{Algoritmi seminumerici}
\begin{itemize}
\item
$25$
\item
$25=00011001$
\item
$25=00011001=$FFFTTFFT
\end{itemize}
\end{block}
\begin{block}{Operazioni bit-level}
\alert{Or}: $x\lor y$, \alert{And}: $x\land y$, \alert{Xor}: $x\oplus y$\\
\alert{Left Shift}: $x << k$, \alert{Right Shift}: $x >> k$,
\begin{itemize}
\item
Tutte bitwise
\item
Tutte in hardware
\end{itemize}
\end{block}
\end{frame}

\begin{frame}
\frametitle{D\"om\"olki / Baeza-Yates, Gonnet}
\begin{block}{Matrice $M$}
$M(i,j)=1$ sse $P[:i]=T[j-i+1:j]$\\
$0\le i\le m$, $0\le j\le n$
\end{block}
\begin{block}{Occorrenza di $P$ in $T$}
$M(m,\cdot)=1$
\end{block}
\begin{itemize}
\item
$M(0,\cdot)=1$, $M(\cdot,0)=0$
\item
\alert{$M(i,j)=1$} sse $M(i-1, j-1)=1$ AND $P[i]=T[j]$
\end{itemize}
\end{frame}

\begin{frame}
\frametitle{Esempio}
\begin{block}{Esempio}
$T$=abracadabra\\
$P$=abr
\end{block}

\begin{center}
\begin{tabular}[l]{ll}
%\hline{1}
10010101001\\
01000000100\\
00100000010&$\leftarrow$ \alert{occorrenze}\\%\hline
\end{tabular}
\end{center}

\begin{block}{Matrice $M$}
1 colonna = 1 numero
\end{block}
\end{frame}

\begin{frame}[fragile]
\frametitle{Colonne}
$U[\sigma]$ = array di bit dove $U[\sigma,i]=1$ sse $P[i]=\sigma$

\begin{block}{$C[j]$ da $C[j-1]$}
\begin{itemize}
\item
Right shift di $C[j-1]$
\item
$1$ in prima posizione
\item
AND con $U[T[j]]$
\item
\alert{$\omega$}: word size
\item
$C[j] = \left( \left(C[j-1] >> 1 \right) \  | \  \left(1 << (\omega -1) \right) \right)\ \&\  U[T[j]]$;
\end{itemize}
\end{block}
\end{frame}

\begin{frame}[fragile]
\frametitle{Note}
\begin{itemize}
\item
Tempo $O(n)$ se $m\le \omega$
\item
Tempo $O(nm)$
\item
No condizioni
\item
$\omega < m\le 2\omega$?
\end{itemize}
\end{frame}

%%% Local Variables:
%%% TeX-PDF-mode: t
%%% TeX-master: "01-bit-parallel-video"
%%% buffer-file-coding-system: utf-8
%%% End:

